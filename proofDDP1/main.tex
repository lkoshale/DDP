\documentclass{article}
\usepackage[utf8]{inputenc}
\usepackage{hyperref}

\title{A* algorithm}
% \author{lokesh koshale}
\date{February 2019}

\usepackage{natbib}
\usepackage{graphicx}

\begin{document}

%\maketitle

\section{A* Algorithm}
For detailed algorithm please refer to the appendix.
\subsection{Proof}
$h(x) $ is heuristics value for node $x$\\
$g(x) $ is distance from source to node $x$.\\
$C(x) = h(x) + g(x). $ we will refer C(x) as cost value of node $x$.\\
$P(x) $ is set of all the parents of $x$.\\
\\
\textit{Claim:} At any step t of the algorithm if node u with cost value $C(u)$ is picked then we have already computed cost value for all nodes $n$ whose (i) $C(n) < C(u)$ and  (ii) $\exists \hspace{2mm} p \hspace{1mm} \epsilon \hspace{1mm} P(n),\hspace{1mm} C(p) < C(u)$.\\
\\
\\
\textit{Proof:}\\
Let at step t we picked node $d$ with cost value $C(d)$\\
Suppose there exists a node $a$ with cost value $C(a)$ such that $C(a)< C(d)$ and its not computed before t.\\
then,
\\
\\
case(i): if for all parents $p$ of node $a$, either $C(p)$ is greater then $C(d)$ or its cost value is not computed then node $a$ violates part (ii) of claim so our assumptions holds.  
\\
\\
case(ii): If there exists a parent $p$ of node $a$ whose cost value is computed and  $C(p) < C(d) $ then, from step  3 (a) ( see appendix ) we always chose the node with least $C(x)$, which implies $p$ would be picked before $d$. As node $a$ is a successor of $p$ at step 3 (d) (iii) the cost value of node $a$ would be computed.\\
which contradicts our assumption that $C(a)$ is not computed. Also as $C(a)< C(d)$ it would be visited(picked) before $d$.
\\
\\
\\
\textit{Claim:} Follows from first claim\\
If cost value of node $v$ is not computed at step t, then $C(v) \geq C(n) $ or if $C(v) < C(n) $ then $\forall \hspace{1mm} p \hspace{1mm} \epsilon $ P(v) $C(p) \geq C(n) $. \\
\\
\\
\\
\textit{Use Case:} After finding optimal path with cost $C$, If we add an edge from $u->v$ and if cost value of $u$ is not computed then ,\\ (i)the cost value v with edge $u->v$ will be greater then C, \\
(ii)or for all parents of $u$ the cost value is greater then C which implies even if we start from source we will get same optimal path as old.\\
So addition of such edges doesn't affect optimal path, nor the cost values of $v$.

\section{Appendix }
\begin{verbatim}
1.  Initialize the open list

2.  Initialize the closed list
    put the starting node on the open 
    list

3.  while the open list is not empty
    
    a) find the node with the least C on 
       the open list, call it "q"

    b) pop q off the open list
  
    d) for each successor of q
        i) if successor is the goal, stop search
         
        ii) successor.g = q.g + distance between successor and q
            successor.h = distance from goal to successor
          
        iii) if successor.C <= successor.g + successor.h or successor.C is not set
            then successor.C = successor.g + successor.h 
            and if not present in open list add it to the open list.
     end (for loop)
  
    e) push q on the closed list
end (while loop) 
    
\end{verbatim}

\bibliographystyle{plain}
\bibliography{references}
\href{https://www.geeksforgeeks.org/a-search-algorithm/}{Algorithm A*}

\end{document}
